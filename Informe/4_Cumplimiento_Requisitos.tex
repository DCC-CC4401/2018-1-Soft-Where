req
Requisitos generales:
	
	1. El sistema debe permitir que personas naturales creen cuentas en el sistema, solicitando su nombre, RUT, correo electrónico y una contraseña.
		[Página para registrarse donde los datos se guardan en BD]

	2. Todo usuario debe autentificarse para acceder al sistema, usando el correo electrónico que tiene registrado en el sistema y su contraseña.
		[Página para logearse.]

	4. El sistema debe considerar 2 tipos de usuarios: administrador y persona natural (estudiante, funcionario, etc.).
		[]

	5. El sistema debe permitir que personas naturales realicen reservas de artículos.
		[Página para reservar]

	7. El sistema debe llevar un historial de las reservas realizadas por una persona natural, registrándose el usuario, fecha y hora de la reserva.
		[Página del historial]

	8. El sistema debe llevar un historial de los préstamos autorizados por un
	administrador, registrándose el usuario, fecha y hora del préstamo.
		[POST/GET de la DB de cierta info.]

	12.La reserva de cualquier artículo debe realizarse al menos 1 hora antes de la fecha/hora de inicio del préstamo.
		[Quizás desabilitar el préstamo para cosas la siguente hora actual del servidor.]

	14.Solo se puede reservar artículos y espacios en días hábiles, en el horario 09:00 - 18:00.
		[if hora<18 and hora>9 bla...]
	
	15.Cada artículo y espacio debe tener un identificador único en el sistema.
		[ITEM:ID en la DB]

Landing page para personas naturales:
	
	16.El sistema debe permitir buscar artículos.
		[Página de búsqueda]
	
	17.El sistema solo se debe mostrar los artículos que cumplen con los criterios de búsqueda especificados por el usuario. Inicialmente, deben considerar los siguientes criterios: identificador, nombre del artículo, rango de fechas y estado (disponible, en préstamo, en reparación, perdido).
		[Gracias por decirme qué cosas deben estar en la base de datos :)]

	18.El sistema debe mostrar una grilla con los horarios en que están reservadas los espacios administrados por el CEI.
		[La grilla]

Perfil del usuario (vista por el dueño del perfil):

	23.El sistema debe mostrar el nombre, RUT y correo electrónico del usuario.
		[Bienvenido: <USER> (<RUT>) <USERBLALBA>@<dominio>]
	
	25.El sistema debe mostrar si el usuario está habilitado para crear reservas y concretar préstamos.
		[USER_STATE==YES]
	
	26.El sistema debe mostrar un listado de las últimas 10 reservas realizadas por el usuario, ordenados por fecha (más nuevos al inicio de la lista), indicando cuales están pendientes, cuales fueron rechazadas y cuales ya fueron concretadas.
		[Historial junto con reservas acutales]

	28.El sistema deber permitir que el usuario marque una o más reservas pendientes y eliminarlas.
		[El Historial debe ser interactivo]

Ficha de un artículo:
	37.El sistema debe mostrar la siguiente información para un artículo: nombre, foto, texto descriptivo y estado actual (disponible, en préstamo, en reparación, perdido).
		[Modelo para los artículos]

	38.El sistema también debe mostrar un resumen de las fechas en las cuales ha sido reservado el artículo.
		[Historial por artículo]
	
	39.En el caso de los administradores, el sistema debe permitir gestionar la información de un artículo.
		[Herramientas de gestión para admins]

	40.En el caso de una persona natural, el sistema debe permitirle al usuario indicar que quiere reservar el artículo, indicando el rango de fecha y horas del préstamo solicitado.
		[Herraminetas de gestión para users]

Ficha de un espacio:
	51.El sistema debe mostrar una grilla con los horarios en que están reservadas los espacios administrados por el CEI.
		[La grilla de las cosas.]

	54.El sistema debe mostrar un listado de todas las reservas pendientes en el sistema, ordenados por fecha (más nuevos al inicio de la lista).
			

	55.El sistema debe permitir que el administrador marque una o más reservas pendientes, pudiendo cambiar su estado a entregado o rechazado.
		

	56.El sistema debe mostrar un listado de todas los préstamos en el sistema, ordenados por fecha (más nuevos al inicio de la lista). El sistema debe permitir filtrar los préstamos por estado (vigentes, caducados, perdidos).