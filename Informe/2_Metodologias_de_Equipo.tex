El equipo de trabajo utilizó una metodología de trabajo tipo Scrum: esto es, con trabajo diario de avance en "features", divididos para los distintos miembos del equipo, necesarios para la implementación de los requisitos, con revisión constante de los logros conseguidos.
Sin embargo, hay ciertas limitantes que no permiten realizar un trabajo Scrum a cabalidad: no es posible hacer reuniones diarias dada la distancia entre los miembros, así como los horarios de la universidad, además de que no se dispone de los espacios necesarios como para poder llevar una tabla de avances como lo requiere esta metodología de trabajo. Herramientas como Git y Git Flow ayudan a suplir algunas de estas falencias, ya que permiten ir revisando los avances que realizan los otros miembros del equipo y los "features" sobre los que trabajan.
El desarrollo de las implementaciones se realiza utilizando el Framework de Django, que permite levantar el funcionamiento de un servicio web rápidamente con backend basado en Python y frontend basado en html y sintaxis de DTL que separa el funcionamiento del diseño, permitiendo que cada parte pueda trabajar de forma independiente sin tener que repetir información de uno en otro. Otros elementos que se añaden para mejorar el trabajo desarrollado son Bootstrap, para tener un diseño de web más responsivo, y Pillow, para facilitar el manejo de imágenes en la base de datos.
Dada esta facilidad de separar las partes, y las restricciones a las que nos enfrentamos, se decide que los miembos del equipo deberían trabajar en las áreas que consideren son las más fuertes para ellos mismos. De esta forma, se aprovechan las habilidades y los intereses de cada uno de los miembros del equipo. Por ejemplo, Ricardo trabajó principalmente en Backend, en lo referido al modelo de datos y el sistema de login y creación de usuarios; mientras que Pedro trabajó principalmente en el Frontend, pues conocía el funcionamiento de Bootstrap y tiene buenas nociones de diseño. De está forma, cada uno de los miembros del equipo pudo aportar lo mejor de cada uno.
Gracias a esta metodología de trabajo, se pudieron implementar los requerimientos indicados por lo solicitado en las instrucciones de la tarea n° 3: los requisitos 1-2, 4-5, 7-8, 12, 14-18, 23, 25-26, 28, 37-40, 51 y 54-56. En la sección 4 de este informe se especifíca como se satisfacen estos requisitos mediante las interfaces implementadas. En la sección 3 se observara el modelo de datos que subyace a las interfaces implementadas, y como estas se relacionan con los modelos para poder implementar sin tener que re-escribir código. 