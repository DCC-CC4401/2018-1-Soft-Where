Luego de haber realizado la implementación de las interfaces del sistema estudiado durante el curso, se pueden llegar a una serie de conclusiones, en base al trabajo realizado, las herramientas utilizadas y las metodologías aplicadas.\\
En cuanto al trabajo realizado, podemos ver como la organización por requisitos ayuda a organizar los objetivos y la funcionalidad deseada para una aplicación, eliminando las distracciones y las ideas que pueden surgir a medida que se implementan otras funcionalidades. De esta forma, el trabajo se hace más organizado, más directo y más sencillo para todos los que participan. Por otro lado, muchas veces los requisitos implican otras funcionalidades, lo cual complica los límites y las responsabilidades de los trabajadores: por ejemplo, si una persona se encuentra encargada de desarrollar modelos, y otra se encuentra encargada de desarrollar un sistema de login, ambas partes se relacionan con los permisos de usuarios y la posibilidad de hacer préstamos. ¿Quién es el responsable de que el usuario pueda, efectivamente, hacer el pedido? \\
Las herramientas utilizadas, en particular Git, ayudan a superar estos problemas y ha hacer que el trabajo sea más armonioso. Si una parte no está resuelta o si falta funcionalidad, los responsables de las partes individuales pueden compartir sus inquietudes y resolver aquello que impide que su código funcione. De la misma forma, Django ayuda a separar las responsabilidades, asegurando que gran parte del sistema pueda funcionar de forma autonoma a las otras (es decir, desarrollar un frontend independiente del desarrollo del backend, pero que asume que el otro estará funcionando para poder tener en "andando" el sistema). \\
Finalmente, la metodología de trabajo tipo Scrum que se utilizó ayuda a acelerar los resultados deseados del trabajo. Poder separar el trabajo no solo en fronend y backend, si no que en "features", aseguraba el avance constante del quehacer de cada uno de los miembros del trabajo.